\documentclass{uniquecv}

\usepackage{fontawesome}

% ----------------------------------------------------------------------------- %

\begin{document}

\name{陈梁子豪}

\medskip

\basicinfo{
  \faPhone ~ (+86) 18011226636
  \textperiodcentered\
  \faEnvelope ~ kiritoclzh@qq.com
  \textperiodcentered\
  \faSuitcase ~ 前端开发 
  \textperiodcentered\
  \faGithub ~ github.com/KiritoKing
}


% ----------------------------------------------------------------------------- %

\section{教育背景}
\dateditem{\textbf{华中科技大学} \quad 计算机科学与技术 \quad 本科}{2020年 -- 2024年}
成绩:GPA 3.69/4.0\quad 英语能力:TOFEL 98 \quad CET4\&6


% ----------------------------------------------------------------------------- %

\section{专业技能}
\smallskip
\textbf{JavaScript, TypeScript, React}, C/C++, C\#/.NET, Python, Java, ML, Linux, Electron, Vue3
\\
对React与TypeScript项目开发有较多经验,其他Web技术如Vue、Electron也有一定经验;对.NET开发、机器学习等领域略有涉猎。


% ----------------------------------------------------------------------------- %

\section{校内经历}
\dateditem{\textbf{国光智能存储实验室}}{2022年12月 -- 2023年6月}
\smallskip

% ----------------------------------------------------------------------------- %

\section{企业实习}
\datedproject{立鼎石油科技(成都)}{AI平台开发}{2023年1月 -- 2023年3月}
\datedproject{金山办公(武汉)}{Web前端开发}{2023年3月 -- 2023年6月}
\smallskip


% ----------------------------------------------------------------------------- %


\section{项目经历}

% ---
\datedproject{React网易云音乐}{个人项目}{2021年09月 -- 2021年12月}
\textit{前端}
\vspace{0.4ex}
 
利用React框架构建前端和Express构建后端,实现了全功能网易云播放器;
使用\textbf{函数式组件语法}封装组件,采用\textbf{自定义Hook和Redux}来实现全局状态管理。
脚手架及所有样式和组件均自己搭建,并基本仿照实现了PC客户端的样式。

% ---
\datedproject{EDM转译工具}{实习项目}{2023年04月 -- 2023年06月}
\textit{前端,工具链,编译器}
\vspace{0.4ex}

利用HTML-AST和CSS-AST解析与修改实现HTML语法转译,如将邮件客户端不支持的HTML5语法转为table布局、
将自适应布局转换为固定媒体查询等。基于该CLI工具打造一个配套的EDM工具链,支持额外特性如tailwind css、
web-component等。
\\
该项目后续将作为另一个HTML全链路设计项目的后端,其项目还在推进中,主要设计者仍是本人。

% ---
% \datedproject{视频自动打码客户端}{团队项目}{2022年10月 - 至今}
% \vspace{0.4ex}

% 基于预训练CV模型实现视频的自动实体识别并进行处理,实现自动打码、风格化处理等功能。
% 本人在项目中主要负责前端、多平台部署和API对接的工作。
% \par 前端采用React进行构建,并在桌面端通过IPC,其他端通过RPC机制实现后端接口调用,前端可以使用不同工具部署到全平台,
% 后端可以部署到本机或云服务器。
% 前端接收后端返回的数据后通过canvas实时渲染画面和进度,并流式地向后端传递所需数据。


% ---
\datedproject{\small{.NET通用AI模型平台}}{实习项目}{2023年1月 - 至今}
\textit{.NET, 机器学习}
\vspace{0.4ex}

在.NET平台提供一套通用的人工智能平台的实现,为所有模型提供统一的抽象,
并在平台内提供Python、TorchLib、Tensorflow等常用的运行环境,可以方便地导入并运行外部模型;
此外还提供了使用C\#编写本地模型的能力,也可以使用Python作为脚本语言调用本地能力。

% % ---

% \datedproject{遗传决策树模型}{比赛项目}{2022年2月}
% \textit{遗传算法、决策树、Python}
% \vspace{0.4ex}

% 未使用sklearn实现的遗传算法迭代决策树系统,训练后可输入当天的数据,得到当天的交易建议。
% 决策树结点包含若干决策因子和一个决策类型参数,从根节点到叶子节点的路径决定一个决策操作;
% 用比特串给决策树编码作为基因,对比特串进行基因操作得到随机子代进行筛选,迭代过程中提供矫正接口,可以使用其他信息进行矫正优化

% % ---

% \datedproject{React问卷管理客户端}{个人项目}{2023年3月}
% \textit{客户端、前端}
% \vspace{0.4ex}

% 使用React和Electron构建的本地问卷管理系统,其中React构建界面,Electron负责沟通系统API(如文件关联等)。
% 问卷和结果均以JSON文件存储和传输,提供问卷编辑、结果管理和导出等功能。

% % ---

% \datedproject{后台管理系统}{个人项目}{2022年12月}
% \textit{前端、全栈}
% \vspace{0.4ex}

% 前后端分离,使用React+antd构建前端页面,koa.js构建后端服务器;
% 实现了登录和账号权限系统,数据以JSON格式存储在服务端,以restful-api形式供前端调用。

% ----------------------------------------------------------------------------- %

\section{获奖情况}
\datedaward{H奖(二等奖)}{MCM 全美大学生数学建模大赛} {2023年5月}
\datedaward{S奖}{MCM 全美大学生数学建模大赛} {2022年5月}
\datedaward{优秀共青团员}{华中科技大学校团委} {2021年5月}
\medskip


% ----------------------------------------------------------------------------- %

\end{document}

